\chapter{Introducción}\label{chap:introduccion}
El primer capítulo es siempre una introducción. En ella debes resumir de forma esquemática pero suficientemente clara lo esencial de cada una de las partes del trabajo. La lectura de este primer capítulo ha de dar una primera idea clara de lo que se pretendía alcanzar, las conclusiones a las que se ha llegado y del procedimiento seguido.
Como tal, es uno de los capítulos más importantes de la memoria. Las ideas principales a transmitir son la identificación del problema a tratar, la justificación de su importancia, los objetivos generales (a grandes rasgos) y un adelanto de la contribución que esperas hacer. 
En esta introducción se englobarán, también, los siguientes subapartados: 
-	Justificación del trabajo / motivación
-	Planteamiento del problema
-	Estructura de la memoria



  %%%
% Document structure   %
%%%                  %%%
\section{Justificación del trabajo}\label{sec:justificaciontrabajo}
En este apartado se deberá presentar el problema de estudio al que se quiere dar solución y justificar su importancia para la comunidad educativa y científica.  
La lectura de este apartado debe dar una idea clara de las razones, motivos e intereses que han llevado a la elección de este tema. Recuerda que para poder justificar este trabajo debe haber referencias a la investigación previa sobre el tema objeto de estudio, independientemente de que luego se profundice en otros apartados. 
Las siguientes preguntas puedan ayudar a la redacción de este apartado:
	¿Cuál es el problema que quieres tratar?
	¿Cuáles crees que son las causas?
	¿Por qué es relevante el problema?

\section{Planteamiento del problema}\label{sec:planteamiento_problema}
Se debe plantear, de forma breve, el problema / necesidad detectada de la que se parte para proponer la propuesta y la finalidad del TFE. Los objetivos se van a plantear posteriormente, pero en este apartado debe quedar claro qué te planteas con la intervención.
Es necesario que los temas escogidos tengan una vinculación directa con la ingeniería de software, la ingeniería web y/o la seguridad informática y, por tanto, el tema trabajado debe estar en consonancia con la titulación. 
Las siguientes preguntas puedan ayudar a la redacción de este apartado: 
	¿Cómo se podría solucionar el problema?
	¿Qué es lo que se propone? Aquí describes tus objetivos en términos generales (ej. “automatización del ciclo de vida de una herramienta open source para soportar integración y entrega continua”).

\section{Estructura de la memoria}\label{sec:estructura}
Aquí describes brevemente lo que vas a contar en cada uno de los capítulos siguientes. 