\chapter{Contexto y estado del arte}\label{chap:objetivos}
%--- Genereal Objectives ---%
Después de la introducción, se suele describir el contexto de aplicación. Suele ser un capítulo (o dos en ciertos casos) en los que se estudia a fondo el dominio de aplicación, citando numerosas referencias. Debe aportar un buen resumen del conocimiento que ya existe en el campo de los problemas habituales identificados y trabajos relacionados. Es la contextualización y antecedentes del trabajo. 
Es conveniente que revises los estudios y tecnologías actuales., y deberás consultar diferentes fuentes bibliográficas. Hay que tener presente los autores, proyectos o empresas de referencia en la temática del trabajo. Si se ha excluido alguna tecnología relevante, hay que justificar adecuadamente su exclusión. Uno de los objetivos principales de este capítulo es mostrar el avance que supone la propuesta presentada en el TFE frente a trabajos relacionados. La organización específica en secciones dependerá estrechamente del trabajo concreto que vayas a realizar. El capítulo debería concluir con el resumen de conclusiones, resumiendo las principales averiguaciones del estudio y cómo van a afectar al desarrollo específico del trabajo y mostrando la ventaja de tu propuesta frente a trabajos relacionados. Por tanto, deben quedar claras las principales contribuciones del trabajo realizado.
Recuerda que debes referenciar adecuadamente los autores que citas en el texto y que en el aula virtual tienes información sobre cómo referenciar según normativa APA. Es necesario que los trabajos de referencia tengan carácter científico, y, por tanto, que su consulta se realice a través de libros, artículos o documentos de prestigio, así como proyectos o aplicaciones oficiales y de especial relevancia.

\section{Contextualización y antecedentes}\label{sec:contextyantec}


%--- Specific Objectives ---%
\section{Trabajos relacionados}\label{sec:trabajos_relacionados}



\section{Conclusiones del estado del arte}\label{sec:conclusionesSOTA}
