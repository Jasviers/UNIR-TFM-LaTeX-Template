\chapter{Objetivos y metodología de trabajo}\label{chap:estadodelarte}
Este bloque es el puente entre el estudio del dominio y la contribución a realizar. Según el tipo concreto de trabajo, el bloque se puede organizar de distintas formas, pero los siguientes elementos deberían estar presentes con mayor o menor detalle.
\section{Objetivo general}\label{sec:objgeneral}
Los trabajos se centran en conseguir un impacto concreto, demostrando la efectividad de una tecnología, proponiendo una nueva funcionalidad, aplicando metodologías para un fin con valor añadido en el desarrollo software, o aportando nuevas herramientas tecnológicas. El objetivo por tanto no debe ser sin más “crear una herramienta”, “usar una tecnología” o “comparar dos sistemas”, sino que debe centrarse en conseguir un efecto observable.
\section{Objetivos específicos}\label{sec:objespecificos}
Independientemente del tipo de trabajo, el objetivo general típicamente se dividirá en un conjunto de objetivos más específicos analizables por separado. Suelen ser explicaciones de los diferentes pasos a seguir en la consecución del objetivo general. 
Con los objetivos, has de concretar qué pretendes conseguir. Se formulan con un verbo en infinitivo más el contenido del objeto de estudio. Se pueden utilizar fórmulas verbales, como las siguientes: 
•	Analizar
•	Calcular
•	Clasificar
•	Comparar
•	Conocer
•	Cuantificar
•	Desarrollar
•	Describir
•	Descubrir
•	Determinar
•	Establecer
•	Explorar
•	Identificar
•	Indagar
•	Medir
•	Sintetizar
•	Verificar

\section{Metodología del trabajo}\label{sec:metodologia_trabajo}
De cara a alcanzar los objetivos específicos (y con ellos el objetivo general), será necesario realizar una serie de pasos. La metodología del trabajo debe describir qué pasos se van a dar para alcanzar los objetivos, el porqué de cada paso, qué instrumentos se van a utilizar, cómo se van a analizar los resultados, etc.